\documentclass{book}
\usepackage[utf8]{inputenc}

\title{Navegación autónoma para un vehículo sin conductor usando el simulador Webots}
\author{Luis David Torres Trejo}
\date{October 2021}

\begin{document}
\maketitle
\tableofcontents


\chapter{Introducción}
\section{Motivación}
\section{Planteamiento del problema}
\section{Hipótesis}
\section{Objetivos}
\section{Descripción del documento}

\chapter{Antecedentes}
\section{Vehículos sin conductor}
\section{Simuladores}
\section{Conceptos básicos de visión artificial}
\section{Máquinas de estados finitas}
\section{Redes neuronales artificiales}
\section{Trabajo relacionado}

\chapter{Simulación con Webots}
\section{Conceptos básicos del simulador Webots}
\section{el formato Open Street Maps}
\section{El ambiente de simulación}

\chapter{Detección de carriles por color}
\section{Segmentación por color}
\section{Umbralización adaptable}
\section{Transformada Hough}

\chapter{Detección de carriles con \textit{ANN}}
\section{Arquitectura de la ANN}
\section{Obtención del dataset}
\section{Entrenamiento}

\chapter{Seguimiento de carriles}
\section{Modelo cinemático del vehículo}
\section{Leyes de control}

\chapter{Detección de obstáculos}
\section{El Filtro de Kalman Extendido}
\section{Estimación de velocidad con el EKF}
\section{Máquinas de estado finitas}
\section{Maniobras de rebase}

\chapter{Pruebas y resultados}
\section{Integración mediante la plataforma ROS}
\section{Pruebas de navegación sin obstáculos}
\section{Pruebas de navegación con obstáculos}

\chapter{Discusión}
\section{Conclusiones}
\section{Trabajo futuro}


\end{document}
