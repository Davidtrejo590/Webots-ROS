\chapter{Discusión} \label{cap:discusión}
\section{Conclusiones} \label{sec:conclusiones}

Al termino de este trabajo se desarrolló un sistema de visión artificial capaz de reconocer carriles a partir de imágenes RGB. Con esto se comprobó que las herramientas de detector de bordes de Canny y transformada Hough son bastante efectivas en el desarrollo de sistemas de visión artificial. Este sistema sirvió como imitación del sentido de la vista humana para el vehículo y en base a ello se diseñaron leyes de control que le permitieron al vehículo desplazarse lateral y longitudinalmente sin abandonar su carril. Para que el vehículo tuviera un mejor conocimiento del ambiente se implementó un algoritmo de agrupación (\textit{k-means}) sobre una nube de puntos y así obtener información acerca de los objetos presentes en los escenarios de pruebas. También, se diseñó un filtro de Kalman extendido para estimar posición y velocidad de obstáculos potencialmente peligrosos en el camino. En conjunto, los subsistemas de detección de carril, detección y seguimiento de objetos formaron el sistema de percepción para el vehículo autónomo.

El sistema de decisión del vehículo no tripulado se logró en primer lugar por el sistema de percepción, como consecuencia del reconocimiento del ambiente se pudo desarrollar un sistema de árbitro que lograra elegir el comportamiento que mejor convenga en diferentes situaciones viales. La segunda parte de este sistema fue gracias a la elección e implementación de comportamientos similares (seguir carril, rebasar, mantener distancia) a los que realiza un conductor humano.

Se crearon diferentes escenarios de prueba utilizando el simulador Webots tratando de cubrir una pequeña parte de los múltiples escenarios que se pueden presentar en un ambiente vial cotidiano. Con esto se consiguieron comprobar diferentes hipótesis. Primero, al utilizar el simulador Webots durante todo el proyecto se obtiene un claro ejemplo de que los simuladores son alternativas muy rentables en el desarrollo de tecnologías relacionadas con la conducción autónoma, pues los sensores, vehículos y otros elementos del mobiliario urbano fueron esenciales en el desarrollo de los subsistemas que forman parte de la arquitectura de navegación del vehículo inteligente. Por otro lado, para las diferentes pruebas de navegación el simulador generalmente entregó buenos resultados respecto al control del vehículo, lo cual fue muy importante en la detección de errores y posibles escenarios catastróficos. También, se resolvió el problema de no contar con los elementos físicos necesarios para el desarrollo de estas tecnologías, con el uso del simulador Webots u algún otro, el problema se reduce a explorar, diseñar e implementar conceptos teóricos del mundo de la robótica para generar posibles soluciones que sumen a la conducción autónoma.

Los comportamientos ``Seguir Carril'', ``Rebase'' y ``Mantener distancia'' definidos para el vehículo además de el sistema de árbitro fueron implementados mediante máquinas de estados finitos, en la mayoría de las simulaciones estos comportamientos ofrecieron resultados positivos y se comprobó que las máquinas de estados son funcionales para implementar comportamientos reactivos. Sin embargo, tal como se explicó en el capítulo de pruebas y resultados el comportamiento de ``Rebase'' fue el más irregular durante las pruebas de navegación como consecuencia de realizarlo en lazo abierto. A pesar de esto, los resultados fueron satisfactorios tanto pruebas de conducción autónoma como para descubrir posibles mejoras a implementar en el futuro.

Es importante mencionar que el trabajo realizado con el simulador Webots en el desarrollo de ambientes urbanos y uso de los sensores para instrumentar al vehículo autónomo fueron útiles durante el Torneo Mexicano de Robótica 2022 en su modalidad virtual. Así mismo, el sistema de navegación autónoma desarrollado fue puesto a prueba en la misma competencia, donde se obtuvieron resultados positivos en las pruebas de navegación autónoma sin obstáculos, navegación autónoma con obstáculos con y sin movimiento además de una prueba adicional para estacionamiento autónomo, la cual no se incluye en este trabajo. El desempeño realizado durante esta competición ayudó a confirmar el funcionamiento estable del sistema de navegación en diferentes situaciones.

\section{Trabajo futuro} \label{sec:trabajo_futuro}

En cuanto al trabajo futuro se pretenden mejorar o en su defecto diseñar nuevas leyes de control para movimiento lateral y longitudinal del vehículo autónomo, sin dejar de lado el modelo cinemático del mismo. En específico se requiere de un control más estable para el control de dirección (\textit{steering}) con velocidades más altas. Como se mencionó en el capítulo de pruebas y resultados, el control del vehículo se vio comprometido en curvas debido a la alta velocidad del vehículo, teniendo como resultado no realizar giros en curvas o en casos extremos provocar una volcadura.

Con el fin de obtener mejores resultados en acciones de rebase es necesario mejorar el comportamiento de ``Rebase''. Se esperan obtener características de los vehículos por rebasar como: medidas horizontales y verticales, posición real en lugar de una posición media. Con base en estas características calcular un control de velocidad y dirección más adecuado para el vehículo autónomo y obtener acciones de rebase más estables sin depender de aspectos controlados como se probó en algunas simulaciones. Además, se buscarán mejorar los sistemas de detección y seguimiento de objetos para tomar en cuenta no solo la posición sino también la velocidad de vehículos obstáculo y decidir de mejor manera cuando es posible realizar un rebase manteniendo la seguridad tanto del vehículo no tripulado como la de los demás vehículos. 

Añadir capas para detección de señales viales y reconocimiento de personas son algunas propuestas para mejorar el sistema de visión artificial. Con esto, se busca añadir más detalle a los escenarios de prueba dentro del simulador y como consecuencia obtener resultados cada vez más cercanos a una conducción autónoma en el mundo vial real.

El trabajo futuro también va enfocado en la búsqueda de complementar las funciones operativas y tácticas del vehículo, algunas funciones estratégicas como selección de destinos y planificación de rutas son tópicos interesantes en el tema de conducción autónoma. Por último, es importante realizar múltiples pruebas y adecuaciones al sistema de navegación con otro tipo de sensores utilizados por vehículos autónomos, por ejemplo; sensores RADAR, GPS e incluso utilizar más cámaras de diferentes tecnologías en puntos estratégicos del vehículo.
