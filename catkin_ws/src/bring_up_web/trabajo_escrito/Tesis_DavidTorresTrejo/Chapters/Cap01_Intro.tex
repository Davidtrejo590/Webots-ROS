\chapter{Introducción}
% Breve introducción
En las últimas décadas se ha notado un amplío aumento de esfuerzos en investigación y desarrollo de tecnologías para vehículos sin conductor. Pues se cree que los vehículos inteligentes cuentan con el potencial suficiente para mejorar la seguridad, calidad y confort durante viajes en carretera e incluso algunos de ellos ya se encuentran en algunas carreteras del mundo. Muchos de los avances en este sector de la robótica se deben a los constantes cambios en el campo de la informática que, con el paso de los años ha crecido a pasos agigantados, reduciendo costos y desarrollando tecnologías más eficientes. 

Empresas dedicadas a tecnologías de la información así como algunas del sector automotriz han abierto sus puertas al descubrimiento, investigación y desarrollo de tecnologías para vehículos autónomos y uso de energías limpias. Google, Tesla, Uber, Toyota, Nvidia, Ford son solo una de las múltiples compañías dedicadas a este sector. Además, diferentes universidades de América, Europa y Asia también han contribuido con múltiples trabajos de investigación relacionados con vehículos autónomos.

Por mencionar uno de los varios objetivos en el desarrollo de vehículos inteligentes se encuentra la disminución de tiempos en tránsito mediante planeación de rutas óptimas y conducción sin distracciones, como consecuencia esto provocará un reducción en emisiones contaminantes y consumo de combustibles fósiles, sin mencionar que algunos vehículos modernos ya cuentan con tecnologías para el uso de energías limpias. También, se prevé que en futuro exista el concepto ``movilidad para todos'', es decir, los vehículos autónomos podrán ser un medio de movilidad personal útil y cómodo para personas que por distintas circunstancias tengan limitaciones físicas.

Desde un punto de vista más técnico se debe de entender que un vehículo autónomo es aquél que puede conducir sin la necesidad de un conductor al frente del volante. Sin embargo, no todos los vehículos en la actualidad son de este tipo, la medida en que un vehículo es considerado autónomo varia desde ser operado por un conductor hasta ser completamente independiente, para ello existe una clasificación que define diferentes niveles de autonomía con base en las características de cada vehículo. Por ejemplo, algunos vehículos cuentan con sistemas de conducción autónoma que permiten detectar carriles y advertir cambios de carril al conductor sin perder el rumbo. Otros, cuentan con sistemas para detección y seguimiento de vehículos que circulan por delante con el fin de mantener una distancia cómoda y segura.

En este sentido, la arquitectura de un sistema de autonomía de vehículos no tripulados suele organizarse en dos partes principales: un sistema de percepción y uno de decisión \cite{paden2016survey}. El sistema de percepción se divide en subsistemas encargados de tareas específicas como: mapeo de carreteras y obstáculos estáticos, detección y seguimiento de obstáculos en movimiento, detección e identificación de señales de tráfico, entro otras. Mientras que el sistema para toma de decisiones se divide en otros subsistemas para tareas de planificación de rutas y caminos, selección de comportamientos, planificación de movimientos, evasión de obstáculos, etc.

Sin embargo, como toda tecnología tiene inconvenientes, por ejemplo; ¿Qué sucedería si un vehículo autónomo choca contra otro igual?, ¿Cuál de ellos tendría la culpa?, ¿Los humanos ya no deberían de aprender a manejar?, ¿Cómo estandarizar las leyes de tránsito?. Estas y muchas otras cuestiones en combinación con fallos técnicos debido a la naturaleza de los vehículos autónomos son limitaciones que requieren de mucho esfuerzo, desarrollo y trabajo tanto de la parte técnica como del lado legal. Por ello, entre más alto y confiable sea el nivel de autonomía de un vehículo no tripulado será más sencillo resolver estas situaciones en el futuro.

% En últimos tiempos se han notado un aumento en esfuerzos de investigación y desarrollo de tecnologías para vehículos sin conductor.

% Estos desarrollos se han impulsado por los recientes avances en el campo de la informático junto con el potencial de transformar el transporte automotriz.


% Algunos sistemas para conducción autónoma como la detección de carril se utiliza para facilitar las advertencias de cambio de carril para el conductor y aumentar el control del rumbo del conductor para mantenerse dentro del carril.

% La detección y seguimiento de vehículos que circulan por delante es utilizada por sistemas de crucero adaptativo (ACC) para mantener una distancia segura y cómoda.

% Dar una referencia de accidentes viales provocados por accidentes viales.

% Uno de los principales objetivos de los vehículos autónomos se centran en ejecutar estrategias para mejorar la seguridad, el confort y optimización energética.

% Los vehículos inteligentes han aumentado sus capacidades para conducción automatizada en entornos controlados.

% Los vehículos autónomos proporcionarán un medio de movilidad personal para personas que puedan conducir debido a limitaciones físicas.

% Los vehículos disminuirán los tiempos de tránsito.

% La medida en que un vehículo es autónomo puede variar desde ser operado por humanos hasta ser completamente autónomo.

\section{Motivación}

La movilidad es un problema común en estos días debido a la alta densidad de vehículos circulando en las grandes ciudades del mundo, problemas como tráfico, estrés y accidentes de carretera son algunos ejemplos. Según datos del INEGI solo en México en el año 2020 se reportaron más de 75 000 personas involucradas en accidentes de tránsito, de las cuales al menos un $5\%$ fallecieron en el accidente y el otro $95\%$ presentó algún tipo de lesión \cite{inegi}. Uno de los principales objetivos de la conducción autónoma es reducir significativamente el número de defunciones ocasionadas por accidentes de carretera, pues los mismos datos indican que gran parte de los accidentes son ocasionados por errores humanos durante la conducción y en muchas de estas ocasiones son terceras personas las que sufren mayor daño y no el conductor.

Un vehículo inteligente puede ayudar a cambiar está situación mediante la toma de mejores decisiones pues estas serían mejor planeadas con base en el entorno y factores que lo rodean, además, no existiría el factor de distracción como ocurre con los conductores humanos. Todo esto sin mencionar el cumplimiento de reglas viales que en muchos de los casos son omitidas por conductores y derivan en pequeños o grandes percances.

Sin embargo, las investigaciones y desarrollos de tecnología inteligente para vehículos autónomos comúnmente requiere gran cantidad de recursos económicos y humanos. En general, el hardware necesario para instrumentar un vehículo sin conductor es costoso y difícil de desarrollar, además los ambientes para pruebas de navegación del vehículo deben ser controlados y los más parecidos a un entorno real para obtener resultados y conclusiones significativas. Por ello, en gran medida el desarrollo de estas tecnologías es realizado por grandes empresas involucradas en tecnologías de la información y automotrices, ya que cuentan con los recursos necesarios para innovar.

Otros trabajos relacionados como los hechos por universidades de diferentes latitudes del mundo son desarrollados a través de simuladores. Los simuladores dedicados al campo de la robótica resultan ser una buena alternativa para solucionar el problema de ausencia de recursos físicos, pues mediante un simulador se pueden desarrollar sistemas de control, visión artificial y navegación autónoma para vehículos no tripulados. Se puede contar con el hardware necesario (simulado) para instrumentar un vehículo, además de recursos para crear distintos escenarios de prueba. Claro que al utilizar simuladores se omiten muchos factores presentes en un entorno urbano real, pero sin duda juegan un papel importante en el desarrollo de nuevas tecnologías para vehículos autónomos.

% \textcolor{red}{
% \begin{itemize}
% \item Los accidentes viales son causados en su mayoría por errores humanos.
% \item Los vehículos sin conductor tienen el potencial de disminuir estos accidentes.
% \item Los problemas de tránsito vehicular se podrían disminuir con vehículos sin conductor pues la toma de decisiones podría ser más planeada.
% \item El desarrollo de vehículos sin conductor requiere de una instrumentación costosa y el hardware en general es caro y difícil de desarrollar.
%   \item Los simuladores son una buena alternativa en el desarrollo de sistemas de visión artificial y navegación autónoma para vehículos sin conductor.
% \end{itemize}
% }

\section{Planteamiento del problema}

Con el propósito de contribuir en la investigación de tecnologías para vehículos autónomos utilizando simuladores es necesario un ambiente simulado con características suficientes para diseñar y desarrollar algoritmos de control, visión artificial y navegación autónoma para automóviles inteligentes. En este sentido, es requerido por el vehículo autónomo diferentes sistemas que le permitan percibir el ambiente. Dentro de estos sistemas se encuentran: un sistema de visión artificial capaz de reconocer carriles, un sistema encargado de detectar y seguir obstáculos en movimiento. 

También, se requiere de un conjunto de comportamientos que ayuden a simular el acto de conducir, estos comportamientos deben ser capaces de realizar seguimiento de carriles, seguimiento de vehículos en situaciones de tránsito vehicular y rebase de otros vehículos. Este conjunto de comportamientos en combinación con un sistema de decisión deben dar como resultado una experiencia cercana a la navegación autónoma real en ambientes controlados.

% \textcolor{red}{
% \begin{itemize}
% \item Se requiere un ambiente simulado con las características suficientes para desarrollar algoritmos de visión artificial y navegación autónoma para vehículos sin conductor.
% \item Se requiere un sistema de visión artificial capaz de reconocer carriles en una vía así como otros vehículos.
% \item Se requiere de un conjunto de comportamientos capaces de realizar seguimiento de carriles y rebase de otros vehículos.
% \end{itemize}
% }

\section{Hipótesis}

El planteamiento del problema y el desarrollo del mismo tiene de base las siguientes hipótesis:
\begin{itemize}
    \item Los simuladores son una buena opción durante el desarrollo de tecnologías para vehículos autónomos.
    \item La transformada Hough y el detector de bordes de Canny son lo suficientemente poderosos para lograr el reconocimiento de carriles en imágenes RGB.
    \item El simulador Webots cuenta con las características necesarias para desarrollar sistemas de visión artificial, control y navegación autónoma para vehículos sin conductor.
    % \item Los algoritmos de agrupación son una buena alternativa para el desarrollo de sistemas dedicados a la detección de objetos.
    \item Las máquinas de estados finitos son apropiadas para el desarrollo de sistemas robóticos basados en el paradigma reactivo.
    \item La navegación autónoma de un vehículo no tripulado en un ambiente controlado se puede conseguir mediante la implementación de comportamientos para seguimiento de carriles, rebase y mantener distancia.
\end{itemize}

% \textcolor{red}{
% \begin{itemize}
% \item El simulador Webots tiene las características necesarias para desarrollar sistemas de visión artificial y Navegación autónoma para vehículos sin conductor.
% \item Se puede lograr el reconocimiento de carriles en la escena empleando segmentación por color y Transformada Hough.
% \item Las máquinas de estados son adecuadas para el desarrollo de comportamientos para seguimiento de carriles y rebase.
% \end{itemize}
% }

\section{Objetivos}

Con base en el planteamiento del problema y las hipótesis anteriormente mencionadas se esperan poder alcanzar los siguientes objetivos al final de este trabajo:
\begin{itemize}
    \item Diseñar y modelar un ambiente vial urbano con las condiciones necesarias para realizar pruebas de navegación autónoma con el simulador Webots.
    \item Utilizar las herramientas de detector de bordes de Canny y transformada Hough a fin de desarrollar un algoritmo para detección de carril a partir de imágenes RGB.
    \item Diseñar y desarrollar un sistema de control para seguimiento de carril.
    \item Establecer e implementar comportamientos para seguimiento de carriles, rebase de obstáculos y mantener distancia.
    \item Integrar todos los módulos desarrollados empleando la plataforma ROS y el simulador Webots.
    \item Realizar pruebas de navegación sin obstáculos y pruebas de navegación con obstáculos tanto estáticos como en movimiento.
\end{itemize}

% \textcolor{red}{
% \begin{itemize}
% \item Diseñar un ambiente urbano adecuado para pruebas de navegación autónoma para el simulador Webots.
% \item Desarrollar un algoritmo de detección de carriles a partir de imágenes RGB.
% \item Desarrollar un control para seguimiento de carril.
% \item Desarrollar comportamientos para evasión de obstáculos.
% \item Integrar todos los módulos empleando el simulador Webots y la plataforma ROS.
% \end{itemize}
% }

\section{Descripción del documento}

Este trabajo cuenta con diferentes capítulos que se adentran en el mundo de la conducción autónoma. Para comenzar, en el capítulo \ref{cap:antecedentes} se presentan antecedentes históricos y definiciones de conceptos básicos que serán utilizados a lo largo del proyecto como: conceptos de vehículos autónomos, simuladores, visión artificial, entre otros. Además, se mencionan algunos trabajos relacionados con vehículos sin conductor. Enseguida, el capítulo \ref{cap:simulación_con_webots} está dedicado al simulador Webots, el cual se utilizó para modelar escenarios y realizar pruebas de navegación autónoma, se describen la características básicas del simulador y se explica como utilizar algunas de ellas. Al final de este capítulo se presentan contribuciones de este trabajo que influyeron en la categoría AutoModelCar del TMR-2022.

El capítulo \ref{cap:seguimiento_de_carriles} trata acerca de la detección de carriles a partir de imágenes RGB, se explican teóricamente las herramientas matemáticas de transformada Hough y detector de bordes de Canny. Además, se realiza una implementación de cómo pueden ser utilizadas estas herramientas para detectar carriles en una imagen. Este capítulo concluye con el diseño e implementación de un par de leyes de control para el movimiento lateral y longitudinal del vehículo con base en su modelo cinemático. A continuación, el capítulo \ref{cap:detección_de_obstáculos} se centra en el proceso de detección de obstáculos, se explica como utilizar un algoritmo de agrupación para identificar grupos, además se da a conocer el concepto de filtro de Kalman y su utilidad en aplicaciones de seguimiento de objetos, esto se ejemplifica con el diseño e implementación de un filtro de Kalman para estimar posición y velocidad de vehículos. El capítulo \ref{cap:detección_de_obstáculos} también da una explicación del por qué es necesario desarrollar un algoritmo de empatado durante el proceso de seguimiento de objetos.

Posteriormente, en el capítulo \ref{cap:comportamientos} se describen brevemente los principales paradigmas de la robótica y se menciona qué paradigma es utilizado por el robot (vehículo autónomo) de este trabajo. Después, se definen y describen los comportamientos que debe realizar el robot en distintas situaciones, igualmente se expone un sistema de decisión para la elección de comportamientos con base en observaciones del ambiente. Por su parte, el capítulo \ref{cap:pruebas_y_resultados} es destinado a la presentación de resultados obtenidos por sistema de navegación bajo pruebas de conducción autónoma sin obstáculos y conducción autónoma con obstáculos estáticos y con movimiento.

El capítulo \ref{cap:discusión} es el término del proyecto, se exponen las conclusiones finales del trabajo teniendo en cuenta las hipótesis y objetivos planteados al principio del mismo. Finalmente, se propone el trabajo futuro por hacer en la búsqueda de mejores resultados en conducción autónoma.

